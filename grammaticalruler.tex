%% LyX 2.3.6.1 created this file.  For more info, see http://www.lyx.org/.
%% Do not edit unless you really know what you are doing.
\documentclass[english]{article}
\usepackage[T1]{fontenc}
\usepackage[latin9]{inputenc}
\usepackage{url}
\usepackage{graphicx}

\makeatletter

%%%%%%%%%%%%%%%%%%%%%%%%%%%%%% LyX specific LaTeX commands.
%% Because html converters don't know tabularnewline
\providecommand{\tabularnewline}{\\}

%%%%%%%%%%%%%%%%%%%%%%%%%%%%%% Textclass specific LaTeX commands.
\newenvironment{lyxcode}
	{\par\begin{list}{}{
		\setlength{\rightmargin}{\leftmargin}
		\setlength{\listparindent}{0pt}% needed for AMS classes
		\raggedright
		\setlength{\itemsep}{0pt}
		\setlength{\parsep}{0pt}
		\normalfont\ttfamily}%
	 \item[]}
	{\end{list}}

\makeatother

\usepackage{babel}
\usepackage{listings}
\renewcommand{\lstlistingname}{Listing}

\begin{document}
\title{Learning functions and classes using rules}
\author{Ioannis G. Tsoulos}
\date{Department of Informatics and Telecommunications, University of Ioannina,
47100 Arta, Greece }
\maketitle
\begin{abstract}
In this paper, a novel method is presented for generating rules for
data classification as well as for regression problems. The proposed
method generates simple rules in a high-level programming language
with the help of Grammatical Evolution. The method does not require
any prior knowledge of the objective problem, the memory it requires
for its execution is constant regardless of the objective problem,
and it can be used to detect any hidden dependencies between the features
of the input problem as well. The proposed method was applied to a
wide range of problems in the relevant literature and comparative
results against other machine learning techniques are presented in
this paper.
\end{abstract}

\section{Introduction }

A variety of common problems from research areas can be considered
as classification or regression problems, such as problems from physics
\cite{fc_physics1,nnphysics1,nnphysics2,nnphysics3}, chemistry \cite{fc_chem1,fc_chem2,fc_chem3},
economics \cite{fc_econ1,fc_econ2}, pollution \cite{fc_pollution1,fc_pollution2,fc_pollution3},
medicine \cite{nnmed1,nnmed2} etc. These problems are usually tackled
by learning models such as Artificial Neural Networks \cite{nn1,nn2},
Radial Basis Function (RBF) networks \cite{rbf1,rbf2}, Support Vector
Machines (SVM) \cite{svm}, etc. A review of the methods used in classification
can be found in the work of Kotsiantis et al \cite{class_review}.
Also, a discussion on how the neural networks perform on regression
datasets is given in \cite{nn_review_regression}. In most cases,
these learning models contain a vector of parameters $\overrightarrow{w}$
used to minimize the quantity: 

\begin{equation}
E\left(N\left(\overrightarrow{x},\overrightarrow{w}\right)\right)=\sum_{i=1}^{M}\left(N\left(\overrightarrow{x}_{i},\overrightarrow{w}\right)-y_{i}\right)^{2}\label{eq:eq1}
\end{equation}
The set $\left(\overrightarrow{x_{i}},t_{i}\right),\ i=1,...,M$ is
the so - called train set, with $t_{i}$ being the actual output for
the pattern $\overrightarrow{x_{i}}$. The model is denoted as a function
$N(\overrightarrow{x},\overrightarrow{w})$. The minimization of equation
\ref{eq:eq1} has been performed with a variety of optimization methods
in the relevant literature, such as Back Propagation method \cite{bpnn,bpnn2},
the RPROP method \cite{rpropnn,rpropnn3,rpropnn2}, Quasi Newton methods
\cite{quasinn,quasinn2}, Particle Swarm Optimization \cite{psonn,psonn2,psorbf}
, genetic algorithms \cite{genetic_svm1,genetic_svm2}, simulated
annealing \cite{siman_svm,simann_neural} etc. However, these techniques
face a number of problems, such as:
\begin{enumerate}
\item The overfitting problem. A major problem with learning techniques
is that when applied to unknown data, called also test data, they
produce poor results even if the learning process was successful.
This is because the parameters of the models fit accurately to the
training data but fail to fit into unknown data. This problem is presented
with details in the article by Geman et all \cite{nngeman}. This
problem is tackled by a list of methods such as weight sharing \cite{nnsharing1,weightsharing1},
pruning \cite{nnprunning,rbf_prunning1,rbf_prunning2}, weight decaying
\cite{nndecay1,nndecay2} etc.
\item Long execution time. In most cases, machine learning models have a
number of parameters that is at least proportional to the dimension
of the input problem and many times, as happens for example in artificial
neural networks, it is a multiple of the input dimension. This means
that long execution times of the optimization methods are required,
especially for large datasets. This problem can be solved either by
using parallel optimization techniques that exploit modern parallel
architectures \cite{parallel-multistart,msgpu} or by reducing the
input dimension with feature selection or construction techniques
from existing ones \cite{nnpca1,nnpca2}.
\item It is difficult to explain the solution. In most cases, the generated
machine learning models produce solutions consisting of numerical
series and numerical parameters derived from optimization methods.
For example, an artificial neural network can consist of a sum of
products with several terms, especially in large dimensional problems.
\end{enumerate}
In this paper, an innovative technique is presented, which constructs
rules in a high level programming language to estimate the true output
in a regression or classification problem. The construction of the
rules is done using the Grammatical Evolution technique \cite{ge}.
Grammatical evolution is an evolutionary process that has been applied
with success in many areas, such as music composition \cite{ge_music},
economics \cite{ge_economics}, symbolic regression \cite{key-17},
robot control \cite{ge_robot}, caching algorithms \cite{key-19},
combinatorial optimization \cite{ge_comp} etc.

The proposed method has an advantage over other methods from the relevant
literature, as it does not require prior knowledge of the objective
problem and can be applied without any modification to both regression
and classification problems. Furthermore, since the method uses Grammatical
Evolution, it can discover hidden associations between the features
of the objective problem and furthermore construct rules in a form
that can be understood. Also, the proposed method does not require
any additional use of an optimization method as in traditional machine
learning models, thus avoiding problems of numerical accuracy. The
proposed method requires a fixed amount of memory to store the proposed
solutions, which does not directly depend on the dimension of the
objective problem.

The rest of this article is organized as follows: in section \ref{sec:Method-description}
the proposed method is outlined in detail, in section \ref{sec:Experiments}
the used experimental datasets are presented as well as the comparative
results against the proposed method and other methods from the relative
literature, and finally, in section \ref{sec:Conclusions} a list
of conclusions from the usage of the proposed method are presented.

\section{Method description \label{sec:Method-description}}

\subsection{Usage of Grammatical Evolution \label{subsec:Usage-of-Grammatical}}

Grammatical evolution is an evolutionary procedure that can produce
programs in any programming language. In Grammatical evolution the
chromosomes enclose production rules from a BNF (Backus--Naur form)
grammar\cite{bnf1}. These grammars usually described as a set \textbf{$G=\left(N,T,S,P\right)$},
where
\begin{itemize}
\item \textbf{$N$ }is the set of non-terminal symbols.
\item \textbf{$T$ }is the set of terminal symbols.\textbf{ }For example,
terminal symbols could be the digits or the functional symbols (exp,
log, etc).
\item $S$ is a non-terminal symbol defined as the start symbol of the grammar.
The production initiates from this symbol.
\item \textbf{$P$ }is a set of production rules in the form \textbf{$A\rightarrow a$
}or\textbf{ $A\rightarrow aB,\ A,B\in N,\ a\in T$.}
\end{itemize}
In order for Grammatical Evolution to work, the original grammar is
enhanced by enumerating all production rules. For example, consider
the enhanced grammar of Figure \ref{fig:bnf}. The numbers in parentheses
are the production sequence numbers for each non-terminal symbol.
The constant N is of features for the input data. 

In Grammatical Evolution, the chromosomes are expressed as vectors
of integers. Every element of each chromosome denotes a\textbf{ }production
rule from the provided BNF grammar. The algorithm starts from the
start symbol of the grammar and gradually produces some program string,
by replacing non - terminal symbols with the right hand of the selected
production rule. The selection of the rule has two steps:
\begin{itemize}
\item Take the next element from the chromosome and denote it as V.
\item Select the next production rule according to the the scheme Rule =
V mod R, where R is the number of production rules for the current
non -- terminal symbol. 
\end{itemize}
An example produced by this grammar could be the following:

\begin{lstlisting}[language={C++}]
if(x1>2+sin(x3)) value=10+exp(x2) else value=x1
\end{lstlisting}

\begin{figure}

\caption{The BNF grammar for the proposed method.\label{fig:bnf}}

\begin{lyxcode}
<S>::=~<ifexpr>~value=<expr>~else~value=<expr>~~(0)

<ifexpr>::=~if(<boolexpr>)~value=<expr>~(0)

~~~~~~~~~~~~|<ifexpr>~else~if(<boolexpr>)~value=<expr>~(1)

<boolexpr>::=<expr>~<relop>~<expr>~(0)

~~~~~~~~~~~~|<boolexpr>~<boolop>~<boolexpr>~(1)

<relop>::=~>~(0)

~~~~~~~~~~~|>=~(1)

~~~~~~~~~~~|<~~(2)

~~~~~~~~~~~|<=~(3)

~~~~~~~~~~~|=~~(4)

~~~~~~~~~~~|!=~(5)

<boolop>::=~~\&~(0)

~~~~~~~~~~~|~|~(1)

<expr>~::=~~(<expr>~<op>~<expr>)~~(0)~~~~~~~~~~~~~

~~~~~~~~~~~|~<func>~(~<expr>~)~~~~(1)~~~~~~~~~~~~~

~~~~~~~~~~~|<terminal>~~~~~~~~~~~~(2)~

<op>~::=~~~~~+~~~~~~(0)~~~~~~~~~~~~~

~~~~~~~~~~~|~-~~~~~~(1)~~~~~~~~~~~~~

~~~~~~~~~~~|~{*}~~~~~~(2)~~~~~~~~~~~~~

~~~~~~~~~~~|~/~~~~~~(3)

<func>~::=~~~sin~~(0)~~~~~~~~~~~~~

~~~~~~~~~~~|~cos~~(1)~~~~~~~~~~~~~

~~~~~~~~~~~|exp~~~(2)~~~~~~~~~~~~~

~~~~~~~~~~~|log~~~(3)

<terminal>::=<xlist>~~~~~~~~~~~~~~~~(0)~~~~~~~~~~~~~~~~~~~~~~

~~~~~~~~~~~|<digitlist>.<digitlist>~(1)

<xlist>::=x1~~~~(0)~~~~~~~~~~~~~~

~~~~~~~~~~~|~x2~(1)~~~~~~~~~~~~~~

~~~~~~~~~~~\dots \dots \dots{}~~~~~~~~~~~~~

~~~~~~~~~~~|~xN~(N)

<digitlist>::=<digit>~~~~~~~~~~~~~~~~~~(0)~~~~~~~~~~~~~~~~~

~~~~~~~~~~~|~<digit><digit>~~~~~~~~~~~~(1)

~~~~~~~~~~~|~<digit><digit><digit>~~~~~(2)

<digit>~~::=~0~(0)|~1~(1)~~~~~~~~~~~~~

~~~~~~~~~~~|~2~(2)|~3~(3)~~~~~~~~~~~~~

~~~~~~~~~~~|~4~(4)|~5~(5)~~~~~~~~~~~~~

~~~~~~~~~~~|~6~(6)|~7~(7)~~~~~~~~~~~~~

~~~~~~~~~~~|~8~(8)|~9~(9)

\end{lyxcode}
\end{figure}


\subsection{The main algorithm}

The method has the following steps:
\begin{enumerate}
\item \textbf{Initialization} step. \textbf{}

\begin{enumerate}
\item \textbf{Read} the train data. The train data contains $M$ patterns
as pairs $\left(x_{i},t_{i}\right),\ i=1..M$ where $t_{i}$ is the
actual output for pattern $x_{i}$.
\item \textbf{Set $N_{G}$}, the maximum number of generations.
\item \textbf{Set $N_{C}$}, the number of chromosomes.
\item \textbf{Set $p_{S}$}, the selection rate, with $p_{s}\in[0,1]$.
\item \textbf{Set $p_{M}$}, the mutation rate, with $p_{m}\in[0,1].$
\item \textbf{Initialize} the chromosomes of the population. Every element
of each chromosome is initialized randomly in the range {[}0,255{]}.
\item \textbf{Set} iter=1, as the current number of generations.
\end{enumerate}
\item \textbf{Genetic step}

\begin{enumerate}
\item \textbf{For $i=1,\ldots,N_{C}$ do}

\begin{enumerate}
\item \textbf{Create} using the procedure of subsection \ref{subsec:Usage-of-Grammatical}
an artificial program $c_{i}$ for the corresponding chromosome $g_{i}$. 
\item \textbf{Apply }$c_{i}$ to the training data and \textbf{calculate}
the fitness $f_{i}$ as
\begin{equation}
f_{i}=\sum_{j=1}^{M}\left(c_{i}\left(x_{j}\right)-t_{j}\right)^{2}
\end{equation}
\end{enumerate}
\item \textbf{EndFor}
\item \textbf{Execute} the selection procedure: 
\begin{enumerate}
\item The chromosomes are sorted according to their fitness. 
\item The best\textbf{ $\left(1-p_{s}\right)\times N_{C}$} chromosomes
are transferred without changes to the next generation of the population.
The rest will be replaced by chromosomes that will be produced at
the crossover.
\end{enumerate}
\item \textbf{Execute} the crossover procedure:\textbf{ }For every pair
of produced offsprings, two parents are selected from the current
population by tournament selection.\textbf{ }For every pair $(z,w)$
of parents, two new offsprings $\tilde{z}$ and $\tilde{w}$ are created
with one point crossover. An example of one - point crossover is as
graphically shown in Figure \ref{fig:onepoint}.
\item \textbf{Execute} the mutation procedure. For every element of each
chromosome, select a random number $r\in\left[0,1\right]$ and alter
the corresponding chromosome if $r\le p_{m}$.
\end{enumerate}
\item \textbf{Set} iter=iter+1
\item \textbf{If} $\mbox{iter}\le N_{G}$ goto \textbf{Genetic} Step, 
\item \textbf{Else} obtain $g^{*}$ as the best chromosome in the population.
\item \textbf{Create} the corresponding artificial program $C^{*}$ through
the procedure of subsection \ref{subsec:Usage-of-Grammatical} for
chromosome $g^{*}$
\item \textbf{Apply} $C^{*}$ to the test set of the dataset and report
the results.
\end{enumerate}
\begin{figure}
\caption{Example of one - point crossover.\label{fig:onepoint}}

\centering{}\includegraphics[scale=0.6]{cross}
\end{figure}


\section{Experiments \label{sec:Experiments}}

The software used in the experiments was coded in ANSI C++ using the
freely available QT programming library, which can be downloaded from
the relevant URL \url{https://www.qt.io}. The software used is available
under the GPL license from \url{https://github.com/itsoulos/GrammaticalRuler/}.
Every experiment was conducted 30 times and averages were measured.
In each execution, a different seed for the drand48() random generator
of the C programming language was used. The classification error is
reported for classification datasets on the test set and the average
mean squared error for regression datasets. Also, the well - known
method of 10 - fold cross validation was used for more reliable results.

\subsection{Dataset description }

The was tested on a series of well known datasets from the relevant
literature and the results are compared against other machine learning
techniques. The main repositories used was:
\begin{enumerate}
\item The UCI Machine Learning Repository \url{http://www.ics.uci.edu/~mlearn/MLRepository.html  } 
\item The Keel repository \url{https://sci2s.ugr.es/keel/}\cite{Keel}
\item The Statlib repository \url{ftp://lib.stat.cmu.edu/datasets/index.html }
\item The Kaggle repository \url{https://www.kaggle.com/}
\end{enumerate}
The used classification datasets are:
\begin{enumerate}
\item \textbf{Australian} dataset \cite{australian}, the dataset is related
to credit card applications.
\item \textbf{Balance} dataset \cite{balance}, which is used to predict
psychological states. 
\item \textbf{Dermatology} dataset \cite{dermatology}, which is used for
differential diagnosis of erythemato-squamous diseases. 
\item \textbf{Glass} dataset. This dataset contains glass component analysis
for glass pieces that belong to 6 classes. 
\item \textbf{Hayes Roth} dataset \cite{hayesroth}. This dataset contains
\textbf{5} numeric-valued attributes and 132 patterns. 
\item \textbf{Heart} dataset \cite{heart}, used to detect heart disease. 
\item \textbf{HeartAttack} dataset, a dataset downloaded from \url{https://www.kaggle.com/},
used to predict the chance of heart attack. 
\item \textbf{HouseVotes} dataset \cite{housevotes}, which is about votes
in the U.S. House of Representatives Congressmen. 
\item \textbf{Liverdisorder} dataset \cite{liver}, used for detect liver
disorders in peoples using blood analysis. 
\item \textbf{Ionosphere} dataset, a meteorological dataset used in various
research papers \cite{ion1,ion2}.
\item \textbf{Mammographic} dataset \cite{mammographic}. This dataset be
used to identify the severity (benign or malignant) of a mammographic
mass lesion from BI-RADS attributes and the patient's age. It contains
830 patterns of 5 features each.
\item \textbf{PageBlocks} dataset. The dataset contains blocks of the page
layout of a document that has been detected by a segmentation process.
It has 5473 patterns with 10 features each.
\item \textbf{Parkinsons} dataset,\cite{parkinson} which is created using
a range of biomedical voice measurements from 31 people, 23 with Parkinson's
disease (PD).The dataset has 22 features. 
\item \textbf{Pima} dataset \cite{pima}, used to detect the presence of
diabetes.
\item \textbf{PopFailures} dataset \cite{popfailures}, used in meteorology. 
\item \textbf{Regions2} dataset. It is created from liver biopsy images
of patients with hepatitis C \cite{regions}. From each region in
the acquired images, 18 shape-based and color-based features were
extracted, while it was also annotated form medical experts. The resulting
dataset includes 600 samples belonging into 6 classes.
\item \textbf{Saheart} dataset \cite{saheart}, used to detect heart disease. 
\item \textbf{Sonar} dataset \cite{sonar}. The task here is to discriminate
between sonar signals bounced off a metal cylinder and those bounced
off a roughly cylindrical rock. 
\item \textbf{Student} dataset \cite{Student}, used to predict student's
knowledge level.
\item \textbf{Wine}: dataset, which is related to chemical analysis of wines
\cite{wine1,wine2}.
\item \textbf{Wdbc} dataset \cite{wdbc}, which contains data for breast
tumors. 
\item \textbf{Eeg} dataset. As an real word example, consider an EEG dataset
described in \cite{eeg1,eeg2} is used here. The dataset consists
of five sets (denoted as Z, O, N, F and S) each containing 100 single-channel
EEG segments each having 23.6 sec duration. Sets Z and O have been
taken from surface EEG recordings of five healthy volunteers with
eye open and closed, respectively. Signals in two sets have been measured
in seizure-free intervals from five patients in the epileptogenic
zone (F) and from the hippocampal formation of the opposite hemisphere
of the brain (N). Set S contains seizure activity, selected from all
recording sites exhibiting ictal activity. Sets Z and O have been
recorded extracranially, whereas sets N, F and S have been recorded
intracranially.
\item \textbf{Zoo} dataset \cite{zoo}, where the task is classify animals
in seven predefined classes.
\end{enumerate}
The regression datasets are:
\begin{enumerate}
\item \textbf{Abalone} dataset \cite{abalone}. This data set can be used
to obtain a model to predict the age of abalone from physical measurements. 
\item \textbf{Airfoil }dataset, which is used by the NASA for a series of
aerodynamic and acoustic tests \cite{airfoil}. 
\item \textbf{Baseball} dataset, a dataset to predict the salary of baseball
players.
\item \textbf{BK} dataset, used to estimate the points scored per minute
in a basketball game.
\item \textbf{BL} dataset, which is related with the affects of machine
adjustments on the time to count bolts.
\item \textbf{Concrete} dataset. This dataset is taken from civil engineering\cite{concrete}. 
\item \textbf{Dee} dataset, used to predict the daily average price of the
electricity energy in Spain.
\item \textbf{Diabetes} dataset, a medical dataset.
\item \textbf{FA} dataset, which contains percentage of body fat and ten
body circumference measurements. The goal is to fit body fat to the
other measurements. 
\item \textbf{Housing} dataset. This dataset was taken from the StatLib
library and it is described in \cite{housing}.
\item \textbf{MB} dataset. This dataset is available from Smoothing Methods
in Statistics \cite{MB} and it includes 61 patterns. 
\item \textbf{MORTGAGE} dataset, which contains the Economic data information
of USA.
\item \textbf{NT} dataset \cite{ntdataset}, which is related to the body
temperature measurements.
\item \textbf{PY} dataset \cite{pydataset}, used to learn Quantitative
Structure Activity Relationships (QSARs). 
\item \textbf{Quake} dataset, used to estimate the strength of a earthquake.
\item \textbf{Treasure} dataset, which contains Economic data information
of USA from 01/04/1980 to 02/04/2000 on a weekly basis.
\item \textbf{Wankara} dataset, which contains weather information.
\end{enumerate}

\subsection{Experimental results}

The parameters for the execution of the proposed method are listed
in Table \ref{tab:Experimental-parameters.}. The experimental results
for the classification datasets are listed in Table \ref{tab:results_class}
and for regression datasets in Table \ref{tab:regression_results}.
In both tables, an additional row was added at the end showing the
average classification or regression error for all datasets and it
is denoted by the name AVERAGE. The columns of both tables have the
following meaning:
\begin{enumerate}
\item The column RBF represents the results from an RBF network with 10
parameters.
\item The column MLP stands for the results of a neural network trained
by a genetic algorithm. The number of weights was set to 10. 
\item The column PROPOSED stands for the proposed method.
\end{enumerate}
Judging from the experimental results, it seems that the proposed
method on average outperforms the other machine learning methods.
In classification problems there is a gain of 22\%-30\% and in regression
problems the gain increases to 50\%-70\%. Furthermore, the proposed
method does not require any prior knowledge of the input problem and
the memory space it uses is fixed and independent of the objective
problem. The results produced by the proposed method are in an understandable
form and in which possible dependencies between the characteristics
of the objective problem can be detected. The generated result of
the method could, for example, be used as a function in some high-level
programming language such as the C programming language.

\begin{table}
\caption{Experimental parameters.\label{tab:Experimental-parameters.}}

\centering{}%
\begin{tabular}{|c|c|}
\hline 
PARAMETER & VALUE\tabularnewline
\hline 
\hline 
$N_{C}$ & 500\tabularnewline
\hline 
$N_{G}$ & 2000\tabularnewline
\hline 
$p_{s}$ & 0.1\tabularnewline
\hline 
$p_{m}$ & 0.05\tabularnewline
\hline 
\end{tabular}
\end{table}

\begin{table}
\caption{Experiemental results for classification datasets.\label{tab:results_class}}

\centering{}%
\begin{tabular}{|c|c|c|c|}
\hline 
DATASET & RBF & MLP & PROPOSED\tabularnewline
\hline 
\hline 
AUSTRALIAN & 34.89\% & 32.21\% & 14.84\%\tabularnewline
\hline 
BALANCE & 33.42\% & 8.97\% & 10.40\%\tabularnewline
\hline 
DERMATOLOGY & 62.34\% & 30.58\% & 29.40\%\tabularnewline
\hline 
GLASS & 50.16\% & 60.25\% & 55.19\%\tabularnewline
\hline 
HAYES ROTH & 64.36\% & 56.18\% & 32.08\%\tabularnewline
\hline 
HEART & 31.20\% & 28.34\% & 19.70\%\tabularnewline
\hline 
HEARTATTACK & 32.83\% & 31.54\% & 22.07\%\tabularnewline
\hline 
HOUSEVOTES & 5.99\% & 6.62\% & 3.00\%\tabularnewline
\hline 
IONOSPHERE & 16.22\% & 15.14\% & 11.11\%\tabularnewline
\hline 
LIVERDISORDER & 30.84\% & 31.11\% & 32.32\%\tabularnewline
\hline 
MAMMOGRAPHIC & 21.38\% & 19.88\% & 17.54\%\tabularnewline
\hline 
PARKINSONS & 17.41\% & 18.05\% & 13.11\%\tabularnewline
\hline 
PIMA & 25.75\% & 32.19\% & 24.90\%\tabularnewline
\hline 
POPFAILURES & 7.04\% & 5.94\% & 5.33\%\tabularnewline
\hline 
REGIONS2 & 37.49\% & 29.39\% & 26.87\%\tabularnewline
\hline 
SAHEART & 32.19\% & 34.86\% & 30.97\%\tabularnewline
\hline 
SONAR & 27.85\% & 26.97\% & 28.38\%\tabularnewline
\hline 
STUDENT & 5.74\% & 7.23\% & 5.47\%\tabularnewline
\hline 
WDBC & 7.27\% & 8.56\% & 5.66\%\tabularnewline
\hline 
WINE & 31.41\% & 19.20\% & 11.35\%\tabularnewline
\hline 
Z\_F\_S & 13.16\% & 10.73\% & 7.97\%\tabularnewline
\hline 
ZO\_NF\_S & 9.02\% & 8.41\% & 8.20\%\tabularnewline
\hline 
ZONF\_S & 4.03\% & 2.60\% & 1.98\%\tabularnewline
\hline 
Z\_O\_N\_F\_S & 48.71\% & 65.45\% & 45.98\%\tabularnewline
\hline 
ZOO & 21.77\% & 16.67\% & 9.70\%\tabularnewline
\hline 
\textbf{AVERAGE} & \textbf{26.90\%} & \textbf{24.28\%} & \textbf{18.94\%}\tabularnewline
\hline 
\end{tabular}
\end{table}
\begin{table}
\caption{Experiments for regression datasets.\label{tab:regression_results}}

\centering{}%
\begin{tabular}{|c|c|c|c|}
\hline 
DATASET & RBF & MLP & PROPOSED\tabularnewline
\hline 
\hline 
ABALONE & 7.32 & 7.17 & 5.05\tabularnewline
\hline 
AIRFOIL & 0.05 & 0.003 & 0.002\tabularnewline
\hline 
BASEBALL & 78.89 & 103.60 & 51.18\tabularnewline
\hline 
BK & 0.02 & 0.03 & 0.02\tabularnewline
\hline 
BL & 0.01 & 5.74 & 0.01\tabularnewline
\hline 
CONCRETE & 0.01 & 0.01 & 0.009\tabularnewline
\hline 
DEE & 0.17 & 1.01 & 0.24\tabularnewline
\hline 
DIABETES & 0.49 & 19.86 & 0.67\tabularnewline
\hline 
HOUSING & 57.68 & 43.26 & 23.35\tabularnewline
\hline 
FA & 0.01 & 1.95 & 0.01\tabularnewline
\hline 
MB & 1.91 & 3.39 & 0.06\tabularnewline
\hline 
MORTGAGE & 1.45 & 2.41 & 0.09\tabularnewline
\hline 
NT & 8.15 & 0.05 & 0.08\tabularnewline
\hline 
PY & 0.02 & 105.41 & 0.02\tabularnewline
\hline 
QUAKE & 0.07 & 0.04 & 0.04\tabularnewline
\hline 
TREASURY & 2.02 & 2.93 & 0.10\tabularnewline
\hline 
WANKARA & 0.001 & 0.012 & 0.0004\tabularnewline
\hline 
\textbf{AVERAGE} & \textbf{9.31} & \textbf{17.46} & \textbf{4.76}\tabularnewline
\hline 
\end{tabular}
\end{table}


\section{Conclusions \label{sec:Conclusions}}

An innovative rule generation method for classification and regression
problems was presented in this work. The method constructs rules in
a high-level programming language format and can be applied without
variation to both classification and regression problems. The method
does not have a vector of parameters to be estimated beyond the genetic
algorithm chromosomes. This means that the memory required to run
the method is at the same levels regardless of the size of the input
problem. In addition, it can be used to indirectly select features
from input features but also to find dependencies between existing
features. Furthermore, the method has been applied to a variety of
classification and regression problems, and the results are quite
encouraging. The method is freely available and requires the existence
of the C++ language as well as the freely available QT programming
library. Future extensions of the method may include:
\begin{enumerate}
\item Usage of more advanced stopping rules for the genetic algorithm.
\item Usage of parallel programming techniques to speed up the genetic algorithm.
\item Automatic function generation by Grammatical Evolution.
\end{enumerate}
\begin{thebibliography}{10}
\bibitem{fc_physics1}E.M. Metodiev, B. Nachman, J. Thaler, Classification
without labels: learning from mixed samples in high energy physics.
J. High Energ. Phys. 2017, article number 174, 2017.

\bibitem{nnphysics1}P. Baldi, K. Cranmer, T. Faucett et al, Parameterized
neural networks for high-energy physics, Eur. Phys. J. C \textbf{76},
2016.

\bibitem{nnphysics2}J. J. Valdas and G. Bonham-Carter, Time dependent
neural network models for detecting changes of state in complex processes:
Applications in earth sciences and astronomy, Neural Networks \textbf{19},
pp. 196-207, 2006

\bibitem{nnphysics3}G. Carleo,M. Troyer, Solving the quantum many-body
problem with artificial neural networks, Science \textbf{355}, pp.
602-606, 2017.

\bibitem{fc_chem1}C. G�ler, G. D. Thyne, J. E. McCray, K.A. Turner,
Evaluation of graphical and multivariate statistical methods for classification
of water chemistry data, Hydrogeology Journal \textbf{10}, pp. 455-474,
2002

\bibitem{fc_chem2}E. Byvatov ,U. Fechner ,J. Sadowski , G. Schneider,
Comparison of Support Vector Machine and Artificial Neural Network
Systems for Drug/Nondrug Classification, J. Chem. Inf. Comput. Sci.
\textbf{43}, pp 1882--1889, 2003.

\bibitem{fc_chem3}Kunwar P. Singh, Ankita Basant, Amrita Malik, Gunja
Jain, Artificial neural network modeling of the river water quality---A
case study, Ecological Modelling \textbf{220}, pp. 888-895, 2009.

\bibitem{fc_econ1}I. Kaastra, M. Boyd, Designing a neural network
for forecasting financial and economic time series, Neurocomputing
\textbf{10}, pp. 215-236, 1996. 

\bibitem{fc_econ2}Moshe Leshno, Yishay Spector, Neural network prediction
analysis: The bankruptcy case, Neurocomputing \textbf{10}, pp. 125-147,
1996.

\bibitem{fc_pollution1}A. Astel, S. Tsakovski, V, Simeonov et al.,
Multivariate classification and modeling in surface water pollution
estimation. Anal Bioanal Chem \textbf{390}, pp. 1283--1292, 2008.

\bibitem{fc_pollution2}A. Azid, H. Juahir, M.E. Toriman et al., Prediction
of the Level of Air Pollution Using Principal Component Analysis and
Artificial Neural Network Techniques: a Case Study in Malaysia, Water
Air Soil Pollut \textbf{225}, pp. 2063, 2014.

\bibitem{fc_pollution3}H. Maleki, A. Sorooshian, G. Goudarzi et al.,
Air pollution prediction by using an artificial neural network model,
Clean Techn Environ Policy \textbf{21}, pp. 1341--1352, 2019.

\bibitem{nnmed1}Igor I. Baskin, David Winkler and Igor V. Tetko,
A renaissance of neural networks in drug discovery, Expert Opinion
on Drug Discovery \textbf{11}, pp. 785-795, 2016.

\bibitem{nnmed2}Ronadl Bartzatt, Prediction of Novel Anti-Ebola Virus
Compounds Utilizing Artificial Neural Network (ANN), Chemistry Faculty
Publications \textbf{49}, pp. 16-34, 2018.

\bibitem{nn1}C. Bishop, Neural Networks for Pattern Recognition,
Oxford University Press, 1995.

\bibitem{nn2}G. Cybenko, Approximation by superpositions of a sigmoidal
function, Mathematics of Control Signals and Systems \textbf{2}, pp.
303-314, 1989.

\bibitem{rbf1}J. Park and I. W. Sandberg, Universal Approximation
Using Radial-Basis-Function Networks, Neural Computation 3, pp. 246-257,
1991.

\bibitem{rbf2}H. Yu, T. Xie, S. Paszczynski, B. M. Wilamowski, Advantages
of Radial Basis Function Networks for Dynamic System Design, in IEEE
Transactions on Industrial Electronics \textbf{58}, pp. 5438-5450,
2011.

\bibitem{svm}I. Steinwart, A. Christmann, Support Vector Machines,
Information Science and Statistics, Springer, 2008.

\bibitem{class_review}S.B. Kotsiantis, I.D. Zaharakis, P.E. Pintelas,
Machine learning: a review of classification and combining techniques,
Artif Intell Rev \textbf{26}, pp. 159--190, 2006.

\bibitem{nn_review_regression}M. Adya, F. Collopy, How effective
are neural networks at forecasting and prediction? A review and evaluation.
J. Forecast. \textbf{17}, pp. 481-495, 1998.

\bibitem{bpnn}D.E. Rumelhart, G.E. Hinton and R.J. Williams, Learning
representations by back-propagating errors, Nature \textbf{323}, pp.
533 - 536 , 1986.

\bibitem{bpnn2}T. Chen and S. Zhong, Privacy-Preserving Backpropagation
Neural Network Learning, IEEE Transactions on Neural Networks \textbf{20},
, pp. 1554-1564, 2009.

\bibitem{rpropnn}M. Riedmiller and H. Braun, A Direct Adaptive Method
for Faster Backpropagation Learning: The RPROP algorithm, Proc. of
the IEEE Intl. Conf. on Neural Networks, San Francisco, CA, pp. 586--591,
1993.

\bibitem{rpropnn3}T. Pajchrowski, K. Zawirski and K. Nowopolski,
Neural Speed Controller Trained Online by Means of Modified RPROP
Algorithm, IEEE Transactions on Industrial Informatics \textbf{11},
pp. 560-568, 2015.

\bibitem{rpropnn2}Rinda Parama Satya Hermanto, Suharjito, Diana,
Ariadi Nugroho, Waiting-Time Estimation in Bank Customer Queues using
RPROP Neural Networks, Procedia Computer Science \textbf{ 135}, pp.
35-42, 2018.'

\bibitem{quasinn}B. Robitaille and B. Marcos and M. Veillette and
G. Payre, Modified quasi-Newton methods for training neural networks,
Computers \& Chemical Engineering \textbf{20}, pp. 1133-1140, 1996.

\bibitem{quasinn2}Q. Liu, J. Liu, R. Sang, J. Li, T. Zhang and Q.
Zhang, Fast Neural Network Training on FPGA Using Quasi-Newton Optimization
Method,IEEE Transactions on Very Large Scale Integration (VLSI) Systems
\textbf{26}, pp. 1575-1579, 2018.

\bibitem{psonn}C. Zhang, H. Shao and Y. Li, Particle swarm optimisation
for evolving artificial neural network, IEEE International Conference
on Systems, Man, and Cybernetics, , pp. 2487-2490, 2000.

\bibitem{psonn2}Jianbo Yu, Shijin Wang, Lifeng Xi, Evolving artificial
neural networks using an improved PSO and DPSO \textbf{71}, pp. 1054-1060,
2008.

\bibitem{psorbf}V. Fathi, G.A. Montazer, An improvement in RBF learning
algorithm based on PSO for real time applications, Neurocomputing
111, pp. 169-176, 2013.

\bibitem{genetic_svm1}C.H. Wu, G.H. Tzeng, Y.J. Goo, W.C. Fang, A
real-valued genetic algorithm to optimize the parameters of support
vector machine for predicting bankruptcy, Expert Systems with Applications
\textbf{32}, pp. 397-408, 2007.

\bibitem{genetic_svm2}E. Pourbasheer, S. Riahi, M.R. Ganjali, P.
Norouzi, Application of genetic algorithm-support vector machine (GA-SVM)
for prediction of BK-channels activity, European Journal of Medicinal
Chemistry \textbf{44}, pp. 5023-5028, 2009.

\bibitem{siman_svm}P.F.Pai, W.C. Hong, Support vector machines with
simulated annealing algorithms in electricity load forecasting, Energy
Conversion and Management \textbf{46}, pp. 2669-2688, 2005.

\bibitem{simann_neural}B. Abbasi, H. Mahlooji, Improving response
surface methodology by using artificial neural network and simulated
annealing, Expert Systems with Applications \textbf{39}, pp. 3461-3468,
2012.

\bibitem{nngeman}S. Geman, E. Bienenstock and R. Doursat, Neural
networks and the bias/variance dilemma, Neural Computation \textbf{4},
pp. 1-58, 1992.

\bibitem{nnsharing1}S.J. Nowlan and G.E. Hinton, Simplifying neural
networks by soft weight sharing, Neural Computation 4, pp. 473-493,
1992.

\bibitem{weightsharing1}T. Zhiri, Z. Ruohua, L. Peng, H. Jin, W.
Hao, H. Qijun, C. Sheng, M. Qiming, A hardware friendly unsupervised
memristive neural network with weight sharing mechanism, Neurocomputing
332, pp. 193-202, 2019.

\bibitem{nnprunning}G. Castellano, A. M. Fanelli, M. Pelillo, An
iterative pruning algorithm for feedforward neural networks, IEEE
Transactions on Neural Networks \textbf{8}, pp. 519-531, 1997.

\bibitem{rbf_prunning1}A.L.I. Oliveira, B.J.M. Melo, S.R.L. Meira,
Improving constructive training of RBF networks through selective
pruning and model selection, Neurocomputing \textbf{64}, pp. 537-541,
2005.

\bibitem{rbf_prunning2}G.B. Huang, P. Saratchandran, N. Sundararajan,
A generalized growing and pruning RBF (GGAP-RBF) neural network for
function approximation, IEEE Transactions on Neural Networks \textbf{16},
pp. 57-67, 2005.

\bibitem{nndecay1}N. K. Treadgold and T. D. Gedeon, Simulated annealing
and weight decay in adaptive learning: the SARPROP algorithm,IEEE
Transactions on Neural Networks \textbf{9}, pp. 662-668, 1998.

\bibitem{nndecay2}M. Carvalho and T. B. Ludermir, Particle Swarm
Optimization of Feed-Forward Neural Networks with Weight Decay, 2006
Sixth International Conference on Hybrid Intelligent Systems (HIS'06),
Rio de Janeiro, Brazil, 2006, pp. 5-5.

\bibitem{parallel-multistart}J. Larson and S.M. Wild, Asynchronously
parallel optimization solver for finding multiple minima, Mathematical
Programming Computation \textbf{10}, pp. 303-332, 2018.

\bibitem{msgpu}R. Kamil, S. Reiji, An Efficient GPU Implementation
of a Multi-Start TSP Solver for Large Problem Instances, Proceedings
of the 14th Annual Conference Companion on Genetic and Evolutionary
Computation, pp. 1441-1442, 2012.

\bibitem{nnpca1}Burcu Erkmen, T�lay Y\i ld\i r\i m, Improving classification
performance of sonar targets by applying general regression neural
network with PCA, Expert Systems with Applications \textbf{35}, pp.
472-475, 2008.

\bibitem{nnpca2}Jing Zhou, Aihuang Guo, Branko Celler, Steven Su,
Fault detection and identification spanning multiple processes by
integrating PCA with neural network, Applied Soft Computing \textbf{14},
pp. 4-11, 2014.

\bibitem{ge}M. O'Neill, C. Ryan, Grammatical Evolution, IEEE Trans.
Evolutionary Computation \textbf{5}, pp. 349-358, 2001.

\bibitem{ge_music}Ortega A., S�nchez R., Manuel Alfonseca Moreno,
Automatic composition of music by means of grammatical evolution,
APL '02 Proceedings of the 2002 conference on APL: array processing
languages: lore, problems, and applications Pages 148 - 155.

\bibitem{ge_economics} O\textquoteright Neill M., Brabazon A., Ryan
C., Collins J. J., Evolving Market Index Trading Rules Using Grammatical
Evolution. In: Boers E.J.W. (eds) Applications of Evolutionary Computing.
EvoWorkshops 2001. Lecture Notes in Computer Science, vol 2037. Springer,
Berlin, Heidelberg.

\bibitem{key-17}O\textquoteright Neill M., Ryan C., Grammatical Evolution:
Evolutionary Automatic Programming in a Arbitary Language, Genetic
Programming, vol. 4, Kluwer Academic Publishers, Dordrecht, 2003

\bibitem{ge_robot}Collins J.J., Ryan C., Automatic Generation of
Robot Behaviors using Grammatical Evolution, In Proc. of AROB 2000,
the Fifth International Symposium on Artificial Life and Robotics. 

\bibitem{key-19}O'Neill M., Ryan C., in: K. Miettinen, M.M. Mkel,
P. Neittaanmki, J. Periaux (Eds.), Evolutionary Algorithms in Engineering
and Computer Science, Jyvskyl, Finland, 1999, pp. 127-134

\bibitem{ge_comp}N. R. Sabar, M. Ayob, G. Kendall, R. Qu, Grammatical
Evolution Hyper-Heuristic for Combinatorial Optimization Problems,
IEEE Transactions on Evolutionary Computation \textbf{17}, pp. 840-861,
2013.

\bibitem{bnf1}J. W. Backus. The Syntax and Semantics of the Proposed
International Algebraic Language of the Zurich ACM-GAMM Conference.
Proceedings of the International Conference on Information Processing,
UNESCO, 1959, pp.125-132

\bibitem{Keel}Alcal�-Fdez J., Fernandez A., Luengo J., Derrac J.,
Garc�a S., S�nchez L., Herrera F., KEEL Data-Mining Software Tool:
Data Set Repository, Integration of Algorithms and Experimental Analysis
Framework. Journal of Multiple-Valued Logic and Soft Computing 17,
pp. 255-287, 2011.

\bibitem{australian}J.R. Quinlan, Simplifying Decision Trees. International
Journal of Man-Machine Studies \textbf{27}, pp. 221-234, 1987. 

\bibitem{balance}T. Shultz, D. Mareschal, W. Schmidt, Modeling Cognitive
Development on Balance Scale Phenomena, Machine Learning \textbf{16},
pp. 59-88, 1994.

\bibitem{dermatology}G. Demiroz, H.A. Govenir, N. Ilter, Learning
Differential Diagnosis of Eryhemato-Squamous Diseases using Voting
Feature Intervals, Artificial Intelligence in Medicine. \textbf{13},
pp. 147--165, 1998.

\bibitem{hayesroth}B. Hayes-Roth, B., F. Hayes-Roth. Concept learning
and the recognition and classification of exemplars. Journal of Verbal
Learning and Verbal Behavior \textbf{16}, pp. 321-338, 1977.

\bibitem{heart}I. Kononenko, E. �imec, M. Robnik-�ikonja, Overcoming
the Myopia of Inductive Learning Algorithms with RELIEFF, Applied
Intelligence \textbf{7}, pp. 39--55, 1997.

\bibitem{housevotes}R.M. French, N. Chater, Using noise to compute
error surfaces in connectionist networks: a novel means of reducing
catastrophic forgetting, Neural Comput. \textbf{14}, pp. 1755-1769,
2002.

\bibitem{ion1}J.G. Dy , C.E. Brodley, Feature Selection for Unsupervised
Learning, The Journal of Machine Learning Research \textbf{5}, pp
845--889, 2004.

\bibitem{ion2}S. J. Perantonis, V. Virvilis, Input Feature Extraction
for Multilayered Perceptrons Using Supervised Principal Component
Analysis, Neural Processing Letters \textbf{10}, pp 243--252, 1999.

\bibitem{liver} J. Garcke, M. Griebel, Classification with sparse
grids using simplicial basis functions, Intell. Data Anal. \textbf{6},
pp. 483-502, 2002.

\bibitem{mammographic}M. Elter, R. Schulz-Wendtland, T. Wittenberg,
The prediction of breast cancer biopsy outcomes using two CAD approaches
that both emphasize an intelligible decision process, Med Phys. \textbf{34},
pp. 4164-72, 2007.

\bibitem{parkinson}M.A. Little, P.E. McSharry, E.J. Hunter, Lorraine
O. Ramig (2008), Suitability of dysphonia measurements for telemonitoring
of Parkinson's disease, IEEE Transactions on Biomedical Engineering
\textbf{56}, pp. 1015-1022, 2009.

\bibitem{pima}J.W. Smith, J.E. Everhart, W.C. Dickson, W.C. Knowler,
R.S. Johannes, Using the ADAP learning algorithm to forecast the onset
of diabetes mellitus, In: Proceedings of the Symposium on Computer
Applications and Medical Care IEEE Computer Society Press, pp.261-265,
1988.

\bibitem{popfailures}D.D. Lucas, R. Klein, J. Tannahill, D. Ivanova,
S. Brandon, D. Domyancic, Y. Zhang, Failure analysis of parameter-induced
simulation crashes in climate models, Geoscientific Model Development
\textbf{6}, pp. 1157-1171, 2013.

\bibitem{regions}N. Giannakeas, M.G. Tsipouras, A.T. Tzallas, K.
Kyriakidi, Z.E. Tsianou, P. Manousou, A. Hall, E.C. Karvounis, V.
Tsianos, E. Tsianos, A clustering based method for collagen proportional
area extraction in liver biopsy images (2015) Proceedings of the Annual
International Conference of the IEEE Engineering in Medicine and Biology
Society, EMBS, 2015-November, art. no. 7319047, pp. 3097-3100. 

\bibitem{saheart}T. Hastie, R. Tibshirani, Non-parametric logistic
and proportional odds regression, JRSS-C (Applied Statistics) \textbf{36},
pp. 260--276, 1987.

\bibitem{sonar}R.P. Gorman, T.J. Sejnowski, Analysis of Hidden Units
in a Layered Network Trained to Classify Sonar Targets, Neural Networks
\textbf{1}, pp. 75-89, 1988.

\bibitem{Student}H. T. Kahraman, S. Sagiroglu, I. Colak, Developing
intuitive knowledge classifier and modeling of users' domain dependent
data in web, Knowledge Based Systems \textbf{ 37}, pp. 283-295, 2013.

\bibitem{wine1}M. Raymer, T.E. Doom, L.A. Kuhn, W.F. Punch, Knowledge
discovery in medical and biological datasets using a hybrid Bayes
classifier/evolutionary algorithm. IEEE transactions on systems, man,
and cybernetics. Part B, Cybernetics : a publication of the IEEE Systems,
Man, and Cybernetics Society, \textbf{33} , pp. 802-813, 2003.

\bibitem{wine2}P. Zhong, M. Fukushima, Regularized nonsmooth Newton
method for multi-class support vector machines, Optimization Methods
and Software \textbf{22}, pp. 225-236, 2007.

\bibitem{wdbc}W.H. Wolberg, O.L. Mangasarian, Multisurface method
of pattern separation for medical diagnosis applied to breast cytology,
Proc Natl Acad Sci U S A. \textbf{87}, pp. 9193--9196, 1990.

\bibitem{eeg1}R. G. Andrzejak, K. Lehnertz, F.Mormann, C. Rieke,
P. David, and C. E. Elger, \textquotedblleft Indications of nonlinear
deterministic and finite-dimensional structures in time series of
brain electrical activity: dependence on recording region and brain
state,\textquotedblright{} Physical Review E, vol. 64, no. 6, Article
ID 061907, 8 pages, 2001. 

\bibitem{eeg2}A. T. Tzallas, M. G. Tsipouras, and D. I. Fotiadis,
\textquotedblleft Automatic Seizure Detection Based on Time-Frequency
Analysis and Artificial Neural Networks,\textquotedblright{} Computational
Intelligence and Neuroscience, vol. 2007, Article ID 80510, 13 pages,
2007. doi:10.1155/2007/80510

\bibitem{zoo}M. Koivisto, K. Sood, Exact Bayesian Structure Discovery
in Bayesian Networks, The Journal of Machine Learning Research\textbf{
5}, pp. 549--573, 2004.

\bibitem{abalone}W. J Nash, T.L. Sellers, S.R. Talbot, A.J. Cawthor,
W.B. Ford, The Population Biology of Abalone (\_Haliotis\_ species)
in Tasmania. I. Blacklip Abalone (\_H. rubra\_) from the North Coast
and Islands of Bass Strait, Sea Fisheries Division, Technical Report
No. 48 (ISSN 1034-3288), 1994.

\bibitem{airfoil}T.F. Brooks, D.S. Pope, A.M. Marcolini, Airfoil
self-noise and prediction. Technical report, NASA RP-1218, July 1989. 

\bibitem{concrete}I.Cheng Yeh, Modeling of strength of high performance
concrete using artificial neural networks, Cement and Concrete Research.
\textbf{28}, pp. 1797-1808, 1998. 

\bibitem{housing}D. Harrison and D.L. Rubinfeld, Hedonic prices and
the demand for clean ai, J. Environ. Economics \& Management \textbf{5},
pp. 81-102, 1978.

\bibitem{MB}J.S. Simonoff, Smooting Methods in Statistics, Springer
- Verlag, 1996.

\bibitem{ntdataset}P.A. Mackowiak, S.S. Wasserman, M.M. Levine, A
critical appraisal of 98.6 degrees f, the upper limit of the normal
body temperature, and other legacies of Carl Reinhold August Wunderlich.
J. Amer. Med. Assoc. \textbf{268}, pp. 1578--1580, 1992.

\bibitem{pydataset}R.D. King, S. Muggleton, R. Lewis, M.J.E. Sternberg,
Proc. Nat. Acad. Sci. USA \textbf{89}, pp. 11322--11326, 1992. 
\end{thebibliography}

\end{document}
